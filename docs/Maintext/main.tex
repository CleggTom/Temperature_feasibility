\documentclass{article}
\usepackage[utf8]{inputenc}
\usepackage{outlines}

\title{Temp RM}
\author{Tom}
\date{January 2021}

\begin{document}

\maketitle

\section{Introduction}
\begin{outline}
    \1 Temperature is a universal force in ecological systems
        \2 Temperature has wide ranging effects
        \2 Works across multiple scales of organisation
    \1 of particular interest is how temperature affects ecosystem dynamics 
        \2 discuss studies looking at how temperature affects stability and other measures
    \1 One potentially overlooked aspect is feasibility 
        \2 define feasibility
        \2 discuss how it is oft negliected aspect of ecosystem dynamics
        \2 is important as:
            \3 a prerequisite for other aspects of dynamics (stability, reactivity ect) 
            \3 An way to explore species sorting and its consequences for measures such as species diverstiy ect...  
\end{outline}

Temperature has long been recognised as fundamental driver of many ecological processes, affecting processes occurring at multiple levels. There is a wealth of evidence documenting these effects anywhere from the effects of temperature on individual physiology,  population growth up to community and ecosystems dynamics. These 

\end{document}

\section{Model}
In order to explore the effect of temperature on ecossytem feasability we use the generalised Lotka-Volterra (GLV) system of equaitons: 
\begin{equation}
    \frac{dC_i}{dt}
\end{equation}

which gives the population dyanmics of species in terms .. 